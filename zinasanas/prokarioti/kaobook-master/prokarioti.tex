\begin{refsegment}
    \setchapterimage[6cm]{seaside}
\setchapterpreamble[u]{\margintoc}
\chapter{Šūnu uzbūve}
Dzīvība tiek iedalīta trīs domēnēs — baktēriju, arheju un eikariotu. Šajā nodaļā runāsim par prokariotiem — grupu, kas ietver baktērijas un arhejus.

\section{pip}
Prokarioti ir vienšūnas organismi ar vienkāršu uzbūvi. Tiem parasti ir viena hromosoma, kurai nav galu (tā kā matu gumija).

\begin{figure}[h]
    \centering
    \includegraphics[width=0.7\textwidth]{images/prokariots.pdf}
    \caption[Prokariots]{Tipiska prokariotu šūna~\cite{prokariots}.}
    \label{fig:prokariots}
\end{figure}

\begin{comment}
\begin{marginfigure}[-5.5cm]
	\includegraphics{images/prokariots.pdf}
	\caption[Prokariotu šūna]{Tipiska prokariotu šūna~\cite{prokariots}.}
	%\url{https://commons.wikimedia.org/wiki/File:Mona_Lisa,_by_Leonardo_da_Vinci,_from_C2RMF_retouched.jpg}}
	\labfig{prokariots}
\end{marginfigure}
\end{comment}

% Print bibliography for this chapter only
\printbibliography[heading=subbibliography,title={References},segment=\therefsegment, notkeyword={image}]
\end{refsegment}
